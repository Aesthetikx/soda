\documentclass{beamer}
\usepackage{minted}
\usepackage{fancyvrb}
\usepackage{graphicx}
\usepackage[ampersand]{easylist}

\usetheme{metropolis}

% Custom Fonts
\usepackage[default]{sourcecodepro}
\usepackage[default]{sourcesanspro}
\usepackage{fontawesome}

% Custom Colors
\definecolor{mOrange}{HTML}{F05A2A}
\definecolor{mBlack}{HTML}{1B1C1D}
\definecolor{mWhite}{HTML}{FCFCFC}
\definecolor{mLightGray}{HTML}{F0F0F0}

\setbeamercolor{normal text}{fg=mBlack,bg=mWhite}
\setbeamercolor{alerted text}{fg=mOrange,bg=mWhite}
\setbeamercolor{progress bar}{fg=mOrange}

% Syntax Highlighting
\usemintedstyle{lovelace}

\setbeamertemplate{frame footer}{\textcolor{gray}{\texttt{A2RB}}}

% Custom frames for minted
\newenvironment{slide}[1]
  {\begin{frame}[fragile=singleslide]{#1}%
  }
  {\end{frame}
  }

\newenvironment{langslide}[2]
  {\VerbatimEnvironment
   \begin{slide}{#2}
   \begin{minted}[autogobble]{#1}%
  }
  {\end{minted}
   \end{slide}
  }

\newenvironment{rubyslide}[1]
  {\VerbatimEnvironment
   \begin{langslide}{ruby}{#1}%
  }
  {\end{langslide}
  }

\newenvironment{psqlslide}[1]
  {\VerbatimEnvironment
   \begin{langslide}{perl}{#1}%
  }
  {\end{langslide}
  }

\newenvironment{bashslide}[1]
  {\VerbatimEnvironment
   \begin{langslide}{bash}{#1}%
  }
  {\end{langslide}
  }

%images
\write18{curl -o day2.jpg http://i.imgur.com/PbXhpE1.jpg}
\write18{curl -o later.jpg https://s-media-cache-ak0.pinimg.com/originals/16/ad/bc/16adbc98b3be8701b58437468812831b.jpg}
\write18{curl -o ytho.jpg http://i.imgur.com/knj0Uy6.jpg}
\write18{curl -o idea.jpg http://media.liveauctiongroup.net/i/7919/9725141_2.jpg?v=8CD038C282C3AC0}

\title{Multi-Tenancy on Rails}
\subtitle{
  \textcolor{lightgray}{
    \textit{
      An Ann Arbor Ruby Brigade Feature Length Presentation \\
      \textcolor{mLightGray}{
        {\scriptsize you can go home when the number in the bottom right is 56ish} \\
        {\tiny i know you cant see the numbers now but wait a few slides} \\
      }
    }
  }
}
\date{April 25}
\author{John DeSilva}
\institute{Revela, Inc.}

\begin{document}
  \maketitle

  \section{What is Multi-Tennancy?}

  \begin{frame}{What is Multi-Tennancy?}
    \begin{quote}
      The term ``software multitenancy'' refers to a software architecture in
      which a single instance of software runs on a server and serves multiple
      tenants. A tenant is a group of users who share a common access with
      specific privileges to the software instance.
    \end{quote}

    - Wikipedia
  \end{frame}

  \begin{frame}{What is Multi-Tennancy?}
    In short, we want to scale our customer base by sliding the heroku slider
    to the right instead of creating new heroku apps with a new db per
    customer. \\
    \vspace{1cm}\\
    \textcolor{gray}{\textit{{\tiny if we need to add databases to scale to our
    data size or for customer reasons,\\
    thats fine,\\
    but only one app cluster\\
    (or a multi zone federated cluster)}}}

  \end{frame}

  \section{Types of Tennancy}

  \begin{frame}{`Multi User' Applications}
    Some applications are used by many individual users. Often B2C.
    \begin{itemize}
      \item Twitter
      \item Gmail
      \item Spotify
    \end{itemize}
  \end{frame}

  \begin{frame}{`Tenantless' Applications}
    Many applications do not have tenancy at all. Lots of low touch news, marketing, and cooking sites here.
    \begin{itemize}
      \item Allrecipes
      \item 4chan
      \item Darksky
      \item Buzzfeed
      \item Wikipedia
    \end{itemize}
  \end{frame}

  \begin{frame}{`Organization' Applications}
    Often, enterprise applications have a `shared dataset' amongst users of
    different organizations. B2B SaaS products tend to fall into this bucket.
    \begin{itemize}
      \item Basecamp
      \item Slack
      \item Salesforce
    \end{itemize}
  \end{frame}

  \begin{frame}{`Gray Area' Applications}
    Some applications do not clearly fall into one of the previous three categories.
    \begin{itemize}
      \item Github
      \item Shopify
    \end{itemize}
  \end{frame}

  \begin{bashslide}{Demo 1 / 3}
    $ git checkout no_tennancy
  \end{bashslide}

  %
  % Single Database Scoping
  %
  \section{(1/2): Query Scoping Tennancy}

  \begin{rubyslide}{Straightforward}
    class RecipesController
      def show
        @recipe = Recipe.find(params[:id])
      end
    end
  \end{rubyslide}

  \begin{slide}{Straightforward}
    \includegraphics[scale=0.22]{day2.jpg}
  \end{slide}

  \begin{rubyslide}{Pretty Straightforward}
    class RecipesController
      def show
        @recipe = current_user.recipes
                              .find(params[:id])
      end
    end
  \end{rubyslide}

  \begin{rubyslide}{Kinda Straightforward}
    class RecipesController
      def show
        @recipe = current_user
                  .company
                  .recipes
                  .find(params[:id])
      end
    end
  \end{rubyslide}

  \begin{rubyslide}{Somewhat Straightforward}
    class RecipesController
      before_action :set_recipe # Rails-it-up-a-notch!

      def show; end

      def set_recipe
        @recipe = recipes.find(params[:id])
      end

      delegate :company, to: :current_user
      delegate :recipes, to: :company
    end
  \end{rubyslide}

  \begin{bashslide}{Demo 2 / 3}
    $ git checkout scoping
  \end{bashslide}

  \begin{slide}{}
    \includegraphics[scale=0.6]{later.jpg}
  \end{slide}

  \begin{psqlslide}{\texttt{has\_mariana\_trench}}
    Companies
    \-> Buildings
        \-> Units
            \-> Leases
                \-> Lease Memberships
                    \-> Tenants
                        \-> Payments
  \end{psqlslide}

  \begin{psqlslide}{`Display a list of Payments'}
    select * from payments
    join tenants on tenants.id = payments.tenant_id
    join lms on lms.tenant_idid = tenants.id
    join leases on leases.id = lms.lease_id
    join units on units.id = leases.unit_id
    join buildings on buildings.id = units.building_id
    join companies on companies.id = buildings.company_id
    where companies.id = $1;

    $1 => current_user.company_id;
  \end{psqlslide}

  \begin{slide}{}
    \includegraphics{ytho.jpg}
  \end{slide}

  \begin{slide}{Why not denormalize \texttt{customer\_id?}}
    \begin{easylist}[itemize]
      & The database schema should (probably?) represent your application's data, not your application's data performance needs or access paterns.
      && \textit{This is CRUD, not Twitter}

      & Now you have to have a \texttt{Customer} at arms reach at the point of \texttt{Payment} creation.
      && \textit{You might, but hey - think of the children. Erm, I mean, background jobs and one off scripts}

      & Hey, what else could use a \texttt{customer\_id}?
      && \textit{It will start to get sprinkled on everywhere, I promise}
    \end{easylist}
  \end{slide}

  \begin{slide}{Why not put \texttt{customer\_id?} on every table?}
    I'm pretty sure Citus Data's commercial pg\_shard evolution actually does
    this or something similar
  \end{slide}

  %
  % Schema Tennancy
  %
  %\section{(2/2): Schema-Based Multi Tennancy}
  \begin{slide}{}
    \includegraphics[scale=1.3]{schema.png}
  \end{slide}

  \begin{slide}{Lingo}
    Schema
  \end{slide}

  \begin{slide}{Lingo - Database Schema}
    \alert{Database} Schema \\
    \begin{quote}
      The term `schema' refers to the organization of data as a blueprint of
      how the database is constructed (divided into database tables in the case
      of relational databases). The formal definition of a database schema is a
      set of formulas (sentences) called integrity constraints imposed on a
      database.
    \end{quote}

    - Wikipedia
  \end{slide}

  \begin{rubyslide}{Lingo - Database Schema}
    # db/schema.rb

    ActiveRecord::Schema.define(
      version: 20170424000157
    ) do

      create_table "companies", force: :cascade do |t|
        t.string   "name"

        # ...
  \end{rubyslide}

  \begin{psqlslide}{Lingo - Database Schema}
soda $ psql soda_development

[local] john@soda_development=# \dt

               List of relations

 Schema |         Name         | Type  | Owner 
--------+----------------------+-------+-------
 public | ar_internal_metadata | table | john
 public | companies            | table | john
 public | recipes              | table | john
 public | schema_migrations    | table | john
(4 rows)
  \end{psqlslide}

  \begin{psqlslide}{Lingo - Database Schema}
[local] john@soda_development=# \d recipes

          Table "public.recipes"

    Column    |            Type             | Modifiers
--------------+-----------------------------+----------
 id           | integer                     | not null
 title        | character varying           |
 instructions | character varying           |
 created_at   | timestamp without time zone | not null
 updated_at   | timestamp without time zone | not null
 company_id   | integer                     |
  \end{psqlslide}

  \begin{slide}{Lingo - PostgreSQL Schema}
    \alert{PostgreSQL} Schema \\
    \begin{quote}
      A database contains one or more named schemas, which in turn contain tables. Schemas also contain other kinds of named objects, including data types, functions, and operators. The same object name can be used in different schemas without conflict; for example, both schema1 and myschema can contain tables named mytable. Unlike databases, schemas are not rigidly separated: a user can access objects in any of the schemas in the database they are connected to, if they have privileges to do so.
    \end{quote}
    - https://www.postgresql.org/docs/9.6/static/ddl-schemas.html
  \end{slide}

  \begin{slide}{Lingo - PostgreSQL Schema}
    \textit{There are several reasons why one might want to use schemas:}
    \begin{itemize}
      \item To allow many users to use one database without interfering with each other.
      \item To organize database objects into logical groups to make them more manageable.
      \item Third-party applications can be put into separate schemas so they do not collide with the names of other objects.
    \end{itemize}
  \end{slide}

  \begin{psqlslide}{Lingo - PostgreSQL Schema}
    [local] john@soda_development=# \dn

      List of schemas

      Name  |  Owner
    --------+----------
     public | postgres

    (1 row )
  \end{psqlslide}

  \begin{psqlslide}{Lingo - PostgreSQL Schema}
    [local] john@soda_development=# show search_path;

       search_path
    -----------------
     "$user", public
    (1 row)

    Time: 0.114 ms
  \end{psqlslide}

  \begin{slide}{Mr. Krabs I Have An Idea}
    \includegraphics[scale=0.6]{idea.jpg}
  \end{slide}

  \begin{psqlslide}{Lingo - PostgreSQL Schema}
    [local] john@soda_development=# \dn

      List of schemas

      Name  |  Owner
    --------+----------
     coke   | john
     pepsi  | john
     public | postgres
    (3 rows)
  \end{psqlslide}

  \begin{psqlslide}{Lingo - PostgreSQL Schema}
    [local] john@soda_development=# \dt *.*

            List of relations

 Schema   | Name     | Type  | Owner
----------+----------+-------+-------
 coke     | recipes  | table | john
 pepsi    | recipes  | table | john
 public   | recipes  | table | john

    ...

(74 rows)
  \end{psqlslide}

  \begin{psqlslide}{Public Recipes}
    [local] john@soda_development

    # select title from recipes;

     title 
    -------
    (0 rows)

    Time: 0.405 ms
  \end{psqlslide}

  \begin{psqlslide}{Coke Recipes}
    [local] john@soda_development

    # select title from coke.recipes;

         title     
    ---------------
     Coca-Cola
     Sprite
     Mellow Yellow
    (3 rows)

    Time: 0.355 ms
  \end{psqlslide}

  \begin{psqlslide}{Pepsi Recipes}
    [local] john@soda_development

    # select title from pepsi.recipes;

         title     
    ---------------
     Pepsi
     Mountain Dew
     Mug Root Beer
    (3 rows)

    Time: 0.292 ms
  \end{psqlslide}

  \begin{psqlslide}{Pepsi Recipes, Magically}
    [local] john@soda_development
    # set search path to pepsi;
    # select title from recipes;

         title     
    ---------------
     Pepsi
     Mountain Dew
     Mug Root Beer
    (3 rows)

    Time: 0.292 ms
  \end{psqlslide}

  \section{The \texttt{apartment} Gem}

  \begin{frame}{apartment}
    \faGithub \quad \texttt{influitive/\alert{apartment}}\\
    \vspace{0cm}\\
    \textit{Database multi-tenancy for Rack (and Rails) applications}
  \end{frame}

  \begin{rubyslide}{\texttt{apartment}'s API}
    Apartment::Tenant.create('pepsi')
    # => create schema pepsi;

    Apartment::Tenant.switch!('pepsi')
    # => set search_path to pepsi;

    Apartment::Tenant.drop('pepsi')
    # => drop schema pepsi cascade;
  \end{rubyslide}

  \begin{rubyslide}{Installing  \texttt{apartment}}
    # Gemfile

    gem 'apartment'


    # and then do a

    rails g apartment:install
  \end{rubyslide}

  \begin{rubyslide}{Configuring  \texttt{apartment} - Elevator}
    # config/initializers/apartment.rb

    require 'apartment/elevators/subdomain'

    Rails.application.config.middleware.use \
      'Apartment::Elevators::Subdomain'

    # (this is already the default config)
  \end{rubyslide}

  \begin{psqlslide}{Configuring  \texttt{apartment} - Subdomain Elevator}
    GET pepsi.recipevault.com/recipes/1
      => set search_path to 'pepsi';

    Recipe.find(1)
      => select * from recipes
         where recipes.id = 1;

      (aka)

      => select * from pepsi.recipes
         where recipes.id = 1;
  \end{psqlslide}

  \begin{rubyslide}{Configuring  \texttt{apartment} - Public Models}
    # config/initializers/apartment.rb
    Apartment.configure do |config|


      # Keep companies in the public schema
      config.excluded_models = %w(Company)


    end
  \end{rubyslide}

  \begin{psqlslide}{Configuring  \texttt{apartment} - Public Models}
    Apartment::Tenant.switch!('coke')

    Company.all
      => select * from public.companies;
                       ^^^^^^

    Recipe.all
      => select * from recipes;

      (aka)

      => select * from coke.recipes;
                       ^^^^
  \end{psqlslide}

  \begin{rubyslide}{Configuring  \texttt{apartment} - Schemas}
    # config/initializers/apartment.rb
    Apartment.configure do |config|

      # Find subdomains from Companies
      config.tenant_names = lambda do
        Company.pluck(:subdomain)
      end

    end
  \end{rubyslide}

  \begin{bashslide}{Accessing Local Tenants}
    pepsi.localhost:3000/recipes
  \end{bashslide}

  \begin{bashslide}{Accessing Local Tenants - lvh.me}
    ~ % dig +noall +answer lvh.me @8.8.8.8

    lvh.me.  1544  IN  A  127.0.0.1
  \end{bashslide}

  \begin{slide}{Accessing Local Tenants}
    \texttt{pepsi.lvh.me:3000/recipes} \\
    \\
    \textit{instead of} \\
    \\
    \texttt{localhost:3000/recipes}
  \end{slide}

  \begin{bashslide}{Demo 3 / 3}
    $ git checkout apartment
  \end{bashslide}

  \section{Caveats}

  \begin{rubyslide}{Migrations}
    # Every schema has to be migrated individually.
    # Apartment does this transparently,
    # but it could be problematic with
    # thousands of tenants.
    #
    # It's probably fine.
  \end{rubyslide}

  \begin{rubyslide}{Rails \& Heroku Console}
    # It seems annoying at first
    Company.find_by(subdomain: 'pepsi').activate!
    # but then you realize its worth it
  \end{rubyslide}

  \begin{rubyslide}{Background Jobs}
    # For normal jobs: #solved
    #
    # This adds metadata containing the current
    # tenant to the job payload, then transparently
    # switch!s upon running the job
    gem 'apartment-sidekiq'

  \end{rubyslide}

  \begin{rubyslide}{Background Jobs}
    # For system jobs:
    class MuhJob
      def perform(*args)
        Company.find_each do |company|
          company.activate!

          # Normal job stuff
          Post.old.archive!
        end
      end
    end
  \end{rubyslide}

  \begin{rubyslide}{Action Cable}
    # It's a pretty quick fix.
    class ApplicationCable::Connection < ActionCable::Connection::Base
      identified_by :current_user, :tenant
      #                            ^^^^^^^

      def connect
        self.tenant = request.subdomain
        Apartment::Tenant.switch!(tenant)

        # As usual
        self.current_user = find_current_user
      end
  \end{rubyslide}

  \section{Considerations}

  \begin{frame}{Multi-Tennancy And You}
    Think about what kind of tennancy model your application will have to
    support. This seems obvious, but its actually so obvious that you might
    overlook it and mess up. (It took me three tries to get this right)
  \end{frame}

  \begin{frame}{Multi-Tennancy And You}
    Different types of tennancy do not necessarily correspond to distinct architecture solutions.
    Although Salesforce and Basecamp are both B2B SaaS organizational products, they have very different use cases, business goals, and data requirements.
  \end{frame}

  \begin{frame}{Multi-Tennancy And You}
    Thinking about user behaviour should tell you what kind of tennancy you should support,
    but your data complexity and business requirements should guide you when implementing it.
  \end{frame}

\end{document}
